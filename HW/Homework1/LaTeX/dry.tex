\documentclass{article}
\usepackage{listings}

\usepackage{xcolor}

\definecolor{commentsColor}{rgb}{0.497495, 0.497587, 0.497464}
\definecolor{keywordsColor}{rgb}{0.000000, 0.000000, 0.635294}
\definecolor{stringColor}{rgb}{0.558215, 0.000000, 0.135316}

\lstset{ %
    backgroundcolor=\color{white},   % choose the background color; you must add \usepackage{color} or \usepackage{xcolor}
    basicstyle=\footnotesize,        % the size of the fonts that are used for the code
    breakatwhitespace=false,         % sets if automatic breaks should only happen at whitespace
    breaklines=true,                 % sets automatic line breaking
    captionpos=b,                    % sets the caption-position to bottom
    commentstyle=\color{commentsColor}\textit,    % comment style
    deletekeywords={...},            % if you want to delete keywords from the given language
    escapeinside={\%*}{*)},          % if you want to add LaTeX within your code
    extendedchars=true,              % lets you use non-ASCII characters; for 8-bits encodings only, does not work with UTF-8
    frame=tb,                          % adds a frame around the code
    keepspaces=true,                 % keeps spaces in text, useful for keeping indentation of code (possibly needs columns=flexible)
    keywordstyle=\color{keywordsColor}\bfseries,       % keyword style
    language=Python,                 % the language of the code (can be overrided per snippet)
    otherkeywords={*,...},           % if you want to add more keywords to the set
    numbers=left,                    % where to put the line-numbers; possible values are (none, left, right)
    numbersep=5pt,                   % how far the line-numbers are from the code
    numberstyle=\tiny\color{commentsColor}, % the style that is used for the line-numbers
    rulecolor=\color{black},         % if not set, the frame-color may be changed on line-breaks within not-black text (e.g. comments (green here))
    showspaces=false,                % show spaces everywhere adding particular underscores; it overrides 'showstringspaces'
    showstringspaces=false,          % underline spaces within strings only
    showtabs=false,                  % show tabs within strings adding particular underscores
    stepnumber=1,                    % the step between two line-numbers. If it's 1, each line will be numbered
    stringstyle=\color{stringColor}, % string literal style
    tabsize=2,                     % sets default tabsize to 2 spaces
    title=\lstname,                  % show the filename of files included with \lstinputlisting; also try caption instead of title
    columns=fixed                    % Using fixed column width (for e.g. nice alignment)
}

\title{%
       236319 \\
       Programming Languages \\
       \large HW01 \\
       \small Dry}
\author{Itay Segev and Arad Reder}
\date{April 2023}

\begin{document}

\maketitle

\section{}
\subsection{}
The expression outputs the string "s a f 0 t", and is written in a compressed and obfuscated style. This style is sometimes used in code
golf, where the goal is to write code that is as short as possible, or in obfuscation challenges, where the goal is to make the code difficult
to understand or reverse-engineer.
\subsection{}
The following command:
\begin{lstlisting}
console.log(+!+[]+!+[],
            +!+[]+!+[]+!+[],
            +!+[]+!+[]+!+[]+!+[]+!+[]+!+[],
            +!+[]+!+[]+!+[],+!+[],
            +!+[]+!+[]+!+[]+!+[]+!+[]+!+[]+!+[]+!+[]+!+[])
\end{lstlisting}
outputs \lstinline{2 3 6 3 1 9}.

\section{}
\subsection{}
\begin{lstlisting}[language=Pascal]
fun factorial n : int = if n = 1 then n else (factorial (n - 1)) * n;
\end{lstlisting}

\subsection{}
\begin{lstlisting}[language=Pascal]
fun fibonacci n : int = if n = 1 then 0 else if n = 2 then 1 else fibonacci(n - 1) + fibonacci(n - 2);
\end{lstlisting}

\section{}
\subsection{}
Both Pascal and C programming languages support the concept of enumerations, which allow you to define a set of named constants with
underlying integer values. However, there are some differences in how enumerations are implemented and used in Pascal and C. Here are
some of the main differences:
\begin{description}
    \item[Syntax:] In Pascal, you define an enumeration type using the "type" keyword, followed by the name of the type and the list of possible
        values In C, you define an enumeration type using the "enum" keyword, followed by the name of the type and the list of possible values
        (each separated by a comma).
    \item[Default Values:] In Pascal, the first value in an enumeration list is assigned the value 0 by default, and each subsequent value is assigned
        the next integer value. However, you can explicitly assign values to each member if you want. In C, the first value in an enumeration
        list is assigned the value 0 by default, and each subsequent value is assigned the next integer value. However, you can explicitly assign
        values to each member if you want.
    \item[Scope:] In Pascal, enumeration types are defined at the same level as other types, such as records and arrays. They can be used in any
        part of the program where a type is expected.In C, enumeration types are usually defined in the global scope (outside of any function),
        and can be used in any part of the program where a type is expected. However, you can also define enumeration types within a function,
        in which case they can only be used within that function.
    \item[Type Safety:] In Pascal, enumeration types are distinct types that are not compatible with other types, even if they have the same under-
        lying integer values. In C, enumeration types are represented as integers, and are compatible with other integer types.
\end{description}

\subsection{}
In Pascal, a set type is a special type that represents a collection of distinct values of a given base type. There are some restrictions on set types
in Pascal:
\begin{enumerate}
    \item The base type must be an ordinal type, which means it can be ordered and assigned integer values. This includes integer, char, boolean,
        enumeration types, and subrange types.
    \item The maximum number of elements in a set depends on the size of the base type. For example, if the base type is an enumeration with 8
        values, the maximum number of elements in the set is 256 ($2^8$).
    \item The elements of a set are represented as bits in memory, and the size of a set depends on the number of elements in the base type. For
        example, a set of days of the week requires one byte of memory, since there are 7 elements.
\end{enumerate}
The restrictions on set types exist for several reasons:
\begin{description}
    \item[Efficiency:] By representing sets as bit patterns in memory, Pascal can perform set operations such as union and intersection using
    \item[Type Safety:] By restricting sets to ordinal types, Pascal ensures that set operations are meaningful and well-defined.
    \item[Compiler Compatibility:] The restrictions on sets were designed to ensure that Pascal programs could be compiled and executed on a
        wide range of computer architectures and operating systems.
\end{description}

\end{document}
