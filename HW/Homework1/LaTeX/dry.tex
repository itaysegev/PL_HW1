\documentclass{article}
\usepackage{listings}

\usepackage{xcolor}

\definecolor{commentsColor}{rgb}{0.497495, 0.497587, 0.497464}
\definecolor{keywordsColor}{rgb}{0.000000, 0.000000, 0.635294}
\definecolor{stringColor}{rgb}{0.558215, 0.000000, 0.135316}

\lstset{ %
    backgroundcolor=\color{white},   % choose the background color; you must add \usepackage{color} or \usepackage{xcolor}
    basicstyle=\footnotesize,        % the size of the fonts that are used for the code
    breakatwhitespace=false,         % sets if automatic breaks should only happen at whitespace
    breaklines=true,                 % sets automatic line breaking
    captionpos=b,                    % sets the caption-position to bottom
    commentstyle=\color{commentsColor}\textit,    % comment style
    deletekeywords={...},            % if you want to delete keywords from the given language
    escapeinside={\%*}{*)},          % if you want to add LaTeX within your code
    extendedchars=true,              % lets you use non-ASCII characters; for 8-bits encodings only, does not work with UTF-8
    frame=tb,                          % adds a frame around the code
    keepspaces=true,                 % keeps spaces in text, useful for keeping indentation of code (possibly needs columns=flexible)
    keywordstyle=\color{keywordsColor}\bfseries,       % keyword style
    language=Python,                 % the language of the code (can be overrided per snippet)
    otherkeywords={*,...},           % if you want to add more keywords to the set
    numbers=left,                    % where to put the line-numbers; possible values are (none, left, right)
    numbersep=5pt,                   % how far the line-numbers are from the code
    numberstyle=\tiny\color{commentsColor}, % the style that is used for the line-numbers
    rulecolor=\color{black},         % if not set, the frame-color may be changed on line-breaks within not-black text (e.g. comments (green here))
    showspaces=false,                % show spaces everywhere adding particular underscores; it overrides 'showstringspaces'
    showstringspaces=false,          % underline spaces within strings only
    showtabs=false,                  % show tabs within strings adding particular underscores
    stepnumber=1,                    % the step between two line-numbers. If it's 1, each line will be numbered
    stringstyle=\color{stringColor}, % string literal style
    tabsize=2,                     % sets default tabsize to 2 spaces
    title=\lstname,                  % show the filename of files included with \lstinputlisting; also try caption instead of title
    columns=fixed                    % Using fixed column width (for e.g. nice alignment)
}

\title{%
       236319 \\
       Programming Languages \\
       \large HW01 \\
       \small Dry}
\author{Itay Segev and Arad Reder}
\date{April 2023}

\begin{document}

\maketitle

\section{}
\subsection{}
The expression outputs the string "s a f 0 t", and is written in a compressed and obfuscated style. This style is sometimes used in code
golf, where the goal is to write code that is as short as possible, or in obfuscation challenges, where the goal is to make the code difficult
tnderstand or reverse-engineer.
\subsection{}
The following command:
\begin{lstlisting}
console.log(+!+[]+!+[],
            +!+[]+!+[]+!+[],
            +!+[]+!+[]+!+[]+!+[]+!+[]+!+[],
            +!+[]+!+[]+!+[],
            +!+[],
            +!+[]+!+[]+!+[]+!+[]+!+[]+!+[]+!+[]+!+[]+!+[])
\end{lstlisting}
outputs \lstinline{2 3 6 3 1 9}.

\section{}
\subsection{}
\begin{lstlisting}[language=Pascal]
fun factorial n : int = if n = 1 then n else (factorial (n - 1)) * n;
\end{lstlisting}

\subsection{}
\begin{lstlisting}[language=Pascal]
fun fibonacci n : int = if n = 1 then 0 else if n = 2 then 1 else fibonacci(n - 1) + fibonacci(n - 2);
\end{lstlisting}

\section{}
\subsection{}
Here are some of the differences between enum in C and enum in Pascal:
\begin{itemize}
    \item In C, the enum type is weak, meaning you can treat any enum variable as if it was an integer, and so compare it to integers, increase and decrease it to your liking, etc... In Pascal however, enums are their own type, and so cannot be compared or treated as integers unless explicitly cast to the integer type (using the \lstinline{ord} function).
    \item In C, since enums are treated as integers, they can also be assigned values that don't have any meaning for their enum type. In Pascal, the values of enum variables can only be within the enum's range, meaning you can't increase, decrease, or assign an enum variable outside of the range of values defined in the enum type.
\end{itemize}

\subsection{}
Here are some of the restrictions (and their reasoning) on the \lstinline{set} type in Pascal:
\begin{itemize}
    \item A \lstinline{set} cannot contain multiple elements of the same value. This is due to the fact that sets in Pascal are supposed to mimic mathematical sets, and mathematical sets are defined such that you cannot have the same element multiple times in a set.
        \begin{itemize}
            \item A restriction that stems from this is that sets of any type are limited in size to the size of that type. e.g. boolean sets are restricted to having a maximum of 2 elements (\lstinline{true} and \lstinline{false}).
        \end{itemize}
    \item A \lstinline{set} can only contain members of a single type. Moreover, a set can only contain members of ordinal types with a range between 0 and 255, and so any set can contain at most 256 elements. This restriction stems from the implementation of sets in Pascal, as they are implemented as an array of bits, each indicating whether an element is or isn't a part of the set.
\end{itemize}
\end{document}
